%%%%%%%%%%%%%%%%%%%%%%%%%%%%%%%%%%%%%%%%%%%%%%%%%%%%%%%%%%%%%%%%%%%%%%%%%%%%%%%%
% Medium Length Graduate Curriculum Vitae
% LaTeX Template
% Version 1.2 (3/28/15)
%
% This template has been downloaded from:
% http://www.LaTeXTemplates.com
%
% Original author:
% Rensselaer Polytechnic Institute 
% (http://www.rpi.edu/dept/arc/training/latex/resumes/)
%
% Modified by:
% Daniel L Marks <xleafr@gmail.com> 3/28/2015
% 
% Further modified by:
% Rohan Bavishi <rohan.bavishi95@gmail.com> 9/20/2016
%
% Important note:
% This template requires the simple_style.cls file to be in the same directory 
% as the .tex file. The res.cls file provides the resume style used for 
% structuring the document.
%
%%%%%%%%%%%%%%%%%%%%%%%%%%%%%%%%%%%%%%%%%%%%%%%%%%%%%%%%%%%%%%%%%%%%%%%%%%%%%%%%

%-------------------------------------------------------------------------------
%	PACKAGES AND OTHER DOCUMENT CONFIGURATIONS
%-------------------------------------------------------------------------------

%%%%%%%%%%%%%%%%%%%%%%%%%%%%%%%%%%%%%%%%%%%%%%%%%%%%%%%%%%%%%%%%%%%%%%%%%%%%%%%%
% You can have multiple style options the legal options ones are:
%
%   centered:	the name and address are centered at the top of the page 
%				(default)
%
%   line:		the name is the left with a horizontal line then the address to
%				the right
%
%   overlapped:	the section titles overlap the body text (default)
%
%   margin:		the section titles are to the left of the body text
%		
%   11pt:		use 11 point fonts instead of 10 point fonts
%
%   12pt:		use 12 point fonts instead of 10 point fonts
%
%%%%%%%%%%%%%%%%%%%%%%%%%%%%%%%%%%%%%%%%%%%%%%%%%%%%%%%%%%%%%%%%%%%%%%%%%%%%%%%%

\documentclass[mm]{simple_style}  

% Default font is the helvetica postscript font
\usepackage{helvet}
\usepackage{hyperref}
\usepackage{url}
\usepackage{xcolor}
\hypersetup {
    colorlinks=true,
    linkcolor=colorlink,
    filecolor=magenta,      
    urlcolor=colorlink,
}
\usepackage[left=0.7in, right=2in, top=0.9in]{geometry}
\usepackage{hyperref}
% Increase text height
\textheight=700pt

\begin{document}

%-------------------------------------------------------------------------------
%	NAME AND ADDRESS SECTION
%-------------------------------------------------------------------------------
\name{Xiao Chu}
\qualification{Master of Engineering, ECE, Queen's University}
\emailone{18xc50@queensu.ca}
\phone{+1-613-532-9938}
%-------------------------------------------------------------------------------

\begin{resume}

%-------------------------------------------------------------------------------
%	EDUCATION SECTION
%-------------------------------------------------------------------------------
\section{Education}
\cusemph{Queen's University}, Kingston, Ontario, Canada\\
{\sl Master of Engineering}, Electrical and Computer Engineering, \timeline{Sep' 19 - Dec' 20 (Expected)}\\
\cusemph{GPA: 4.0/4.3}\\
\newline
\cusemph{Wuhan University of Technology}, Wuhan, Hubei, China\\
{\sl Bachelor of Engineering}, Computer Science and Technology, \timeline{Sep' 15 - Jun' 19}\\
\cusemph{GPA: 90.17/100 (3.78/4.0)} Ranked \textbf{10}th over 227 students (top \textbf{5}\%)\\
\sectionline
%-------------------------------------------------------------------------------

%-------------------------------------------------------------------------------
%	RESEARCH SECTION
%-------------------------------------------------------------------------------
\section{Research\\Interests}
\par
Machine Learning, Nature Language Processing\\
Sentiment Analysis, Automatic Text Generation, Text Extraction

%-------------------------------------------------------------------------------
%      PUBLICATIONS 
\sectionline
%-------------------------------------------------------------------------------
%       AWARDS & ACHIEVEMENTS	
%-------------------------------------------------------------------------------
\section{Awards \& Achievements}
\cusemph{Outstanding Graduate}  \timeline{Apr' 2019}, Wuhan University of Technology\\
\newline
\cusemph{The Third-Class Scholarship}\\ TOP 5\% in all students \timeline{Sep' 2018}, Wuhan University of Technology\\
\newline
\cusemph{Academic Excellence Award} \\
Awarded to the outstanding student \\
who got the highest GPA in the whole year \timeline{Apr' 2018}, Wuhan University of Technology\\
\newline
\cusemph{The Second-Class Scholarship}\\ TOP 3\% in all students \timeline{Sep' 2017}, Wuhan University of Technology\\
\newline
\cusemph{The Third-Class Scholarship}\\ TOP 5\% in all students \timeline{Sep' 2016}, Wuhan University of Technology\\


\vspace{-2ex}
\sectionline
%-------------------------------------------------------------------------------

\section{Research\\Projects}
\begin{project}
  \title{SemEval2020 Task5 Modelling Casual Reasoning in Language: Detecting Counterfactuals }
  \supervisor{Supervisor : Dr.Xiaodan Zhu}
  \duration{Sep '19 - Jan '20}
  \description{
	- Constructed the dataset with more than 10,000 text instances. Collected the original data from Internet and removed all non-linguistic information from original data.\\
	- Revised the annotations made by annotators, split the whole dataset into training dataset and test dataset respectively.\\
	- Built a binary classification model as baseline model in subtask 1 using SVM. \\
	- Built a sequence labelling model as baseline model in subtask 2 using CRF. \\
	- Devised a word-level Precision/Recall/F1-score metric and a sentence-level Exact Match metric. Compared with existed metric functions provided by Pytorch, these two metrics could better measure models' performance in subtask 2. \\
	- For more details about this project, please refer to \href{http://competitions.codalab.org/competitions/21691}{SemEval 2020 Task5}.} 
\end{project}
\begin{project}
  \title{Verbal Irony Detection}
  \supervisor{Supervisor : Dr.Xiaodan Zhu}
  \duration{Apr '20 - Now}
  \description{
	- Constructed the dataset which includes more than 1,500 text-audio bi-modality instances.\\
	- Built the audio-modality model based on Transformer, extracted the audio feature using OpenSmile and IS09 feature dataset. This model was a baseline model for audio-modality in this dataset.\\
	- Built the text-modality model based on BERT, This model was a baseline model for text-modality in this dataset.\\
	- Built two multi-modality fusion model as baseline models for multi-modality fusion in this dataset.
	- Devised a bi-linear fusion model based on cartesian product, compared with original fusion model, this model improved 6\% performance in this dataset.\\
	- We plan release our dataset, code on Github and submit our paper to Language Resource and Evaluation this December.\\
	
  }
\end{project}
\sectionline
\section{Computer\\Skills}
\cusemph{Languages}\\ Python (Proficient), C++/C, Matlab, \LaTeX\\
\newline
\cusemph{Deep Learning Framework}\\
{Pytorch (Proficient)}. Read the Python source code of Pytorch 1.7, skilled at building customized neural network and large-scale pre-trained neural network.\\
Tensorflow (Experienced). I have the experience of re-writing the Tensorflow-based code to Pytorch.
\\
\newline
\cusemph{Others}\\
Experienced with Ubuntu(a Linux-based operation system) and Web Development.\\
\vspace{-2ex}
\sectionline
%-------------------------------------------------------------------------------
\section{Machine Learning and Deep Learning\\Background}
\cusemph{Mathematics}\\ 
- Passed undergraduate course "Probability and Mathematics Statistic" and graduate course "Probability, Random Variable, Stochastic Process" (ELEC 861) in full mark.\\
- Read the book "Introduction to Linear Algebra" (Gilbert Strang) to enhance my linear algebra skill.\\

\cusemph{ML\&DL}\\
- Passed graduate course "Machine Learning and Deep Learning" (ELEC 845) at 4.0 GPA.\\
- Read the book "Deep Learning" (Ian Goodfellow et al.) and "Pattern Recognize and Machine Learning" (Christopher Bishop) to enhance my knowledge and understanding in this field.\\
\vspace{-2ex}
\sectionline
%-------------------------------------------------------------------------------
%	Interests
%-------------------------------------------------------------------------------
\section{Extra Interests}
\cusemph{Caligraphy}: Chinese Caligraphy uses writing brush made by weasel's or sheep's hair to write Chinese characters in different forms such as Song, Cao, Xing. It is a traditional art form in China. I practiced caligraphy for 6 years since I was ten year's old. \\
\cusemph{Computer Hardware}: Very interested in computer hardware, assembled a personal computer at very young age independently, helped many classmates choose hardwares (graphic card, cpu, memory, motherboard etc.) and assemble their own computers.
%-------------------------------------------------------------------------------
\end{resume}
\end{document}
